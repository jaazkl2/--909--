\chapter{Methodologi Kajian}n 
\section{Pengenalan}

Bab ini membincangkan metodologi penyelidikan yang digunakan dalam pembangunan aplikasi \textit{AR Alphabets Prasekolah}. Metodologi penyelidikan merujuk kepada pendekatan sistematik yang digunakan oleh penyelidik untuk memperoleh data dan maklumat bagi mencapai objektif kajian (Ismail et al., 2021; Rahman & Zulkifli, 2023). 

Kajian ini menggunakan pendekatan Kajian Reka Bentuk dan Pembangunan (Design and Development Research, DDR) yang sesuai untuk pembangunan aplikasi pendidikan interaktif. Pendekatan ini membolehkan penyelidik membangunkan produk berasaskan keperluan pengguna sebenar serta menilai keberkesanannya secara sistematik (Kamaruddin et al., 2022). 

Bab ini turut menghuraikan reka bentuk kajian, prosedur pelaksanaan, kaedah persampelan, instrumen kajian, kaedah pengumpulan dan analisis data, serta langkah-langkah untuk memastikan kesahan dan kebolehpercayaan kajian mengikut struktur fasa DDR yang dilaksanakan.
\section{Reka Bentuk Kajian}

Kajian ini menggunakan pendekatan Reka Bentuk dan Pembangunan (Design and Development Research, DDR) yang merangkumi kaedah kuantitatif dan kualitatif mengikut fasa pelaksanaan. Kaedah kuantitatif digunakan dalam bentuk soal selidik untuk mengumpul data daripada responden, manakala kaedah kualitatif dilaksanakan melalui temu bual separa berstruktur. Kaedah tinjauan membolehkan data dikumpul dengan pantas dan berkesan, terutamanya apabila melibatkan pelbagai pembolehubah yang boleh dianalisis secara statistik (Zulkifli et al., 2021; Hassan & Rahim, 2022).

Secara keseluruhan, kajian ini berasaskan pendekatan DDR seperti yang digariskan oleh Richey dan Klein (2020), yang sesuai digunakan dalam pembangunan aplikasi pendidikan interaktif seperti \textit{AR Alphabets Prasekolah}. Pendekatan ini membolehkan penyelidik menambah baik proses pembelajaran melalui pembangunan produk yang berasaskan keperluan pengguna sebenar, serta menilai kebolehgunaan aplikasi secara praktikal dan sistematik. DDR juga dikenali sebagai kajian pembangunan (developmental research).

Menurut Richey dan Klein (2020), pendekatan DDR terdiri daripada tiga fasa utama yang sistematik, iaitu:
\begin{itemize}
  \item \textbf{Fasa 1: Analisis Keperluan} – Mengenal pasti keperluan pengguna terhadap aplikasi AR dalam konteks pembelajaran huruf prasekolah.
  \item \textbf{Fasa 2: Reka Bentuk dan Pembangunan} – Membangunkan aplikasi berdasarkan dapatan pakar dan keperluan pengguna menggunakan teknologi seperti Unity3D dan Vuforia SDK.
  \item \textbf{Fasa 3: Penilaian Kepenggunaan} – Menilai kefahaman, minat dan kebolehgunaan aplikasi melalui pemerhatian dan temu bual bersama murid prasekolah.
\end{itemize}

Pendekatan ini membolehkan penyelidik mengaplikasikan pelbagai instrumen dan teknik pengumpulan data secara sistematik mengikut keperluan setiap fasa. Oleh itu, kajian ini dikendalikan secara berperingkat mengikut ketiga-tiga fasa utama DDR yang saling melengkapi.
Kajian ini dilaksanakan melalui tiga fasa utama seperti yang digariskan dalam pendekatan DDR oleh Richey dan Klein (2020). Setiap fasa memainkan peranan penting dalam memastikan pembangunan aplikasi \textit{AR Alphabets Prasekolah} dijalankan secara sistematik dan berasaskan keperluan sebenar pengguna. Fasa-fasa tersebut adalah seperti berikut:

\begin{itemize}
  \item \textbf{Fasa 1: Analisis Keperluan} – Fasa ini bertujuan untuk mengenal pasti keperluan pengguna terhadap aplikasi AR dalam konteks pembelajaran huruf prasekolah. Data dikumpul melalui soal selidik dan temu bual bersama guru prasekolah.
  
  \item \textbf{Fasa 2: Reka Bentuk dan Pembangunan} – Fasa utama kajian ini melibatkan pembangunan aplikasi berdasarkan dapatan fasa pertama. Teknik Fuzzy Delphi digunakan untuk mendapatkan kesepakatan pakar dalam menentukan elemen penting yang perlu dimasukkan ke dalam aplikasi.
  
  \item \textbf{Fasa 3: Penilaian Kepenggunaan} – Fasa terakhir ini menilai kebolehgunaan aplikasi melalui temu bual dan pemerhatian terhadap murid prasekolah. Penilaian ini memberi gambaran tentang kefahaman, minat, dan keberkesanan aplikasi dalam menyampaikan kandungan pembelajaran.
\end{itemize}

Secara ringkasnya, kajian ini dijalankan mengikut tiga fasa utama DDR, iaitu analisis keperluan, reka bentuk dan pembangunan, serta penilaian kepenggunaan. Jadual~\ref{jadual:kaedahDDR} menunjukkan ringkasan kaedah kajian yang dijalankan berdasarkan setiap fasa, yang diadaptasikan daripada buku \textit{Design and Developmental Research: Emergent Trends in Educational Research} (Richey \& Klein, 2020).
\section{Kerangka Prosedur Kajian}

Bagi memastikan kajian ini dilaksanakan secara teratur dan sistematik, satu kerangka prosedur telah dibina untuk menggambarkan aliran pelaksanaan kajian berdasarkan pendekatan Reka Bentuk dan Pembangunan (Design and Development Research, DDR). Kajian ini dibahagikan kepada tiga fasa utama, iaitu fasa analisis keperluan, fasa reka bentuk dan pembangunan aplikasi, serta fasa penilaian kepenggunaan aplikasi. Aliran pelaksanaan kajian ini ditunjukkan dalam Rajah~\ref{rajah:prosedurKajian}.

\begin{table}[H]
\centering
\caption{Kaedah Kajian Mengikut Fasa DDR}
\label{jadual:kaedahDDR}
\begin{tabular}{|p{4cm}|p{10cm}|}
\hline
\textbf{Fasa Kajian} & \textbf{Kaedah yang Digunakan} \\
\hline
Fasa 1: Analisis Keperluan & Soal selidik untuk mengenal pasti keperluan pengguna terhadap aplikasi AR. Analisis data dilakukan berdasarkan skor min dan peratusan. \\
\hline
Fasa 2: Reka Bentuk dan Pembangunan Aplikasi & Kajian literatur dan kaedah Fuzzy Delphi digunakan untuk mendapatkan kesepakatan pakar dalam menentukan elemen penting aplikasi. \\
\hline
Fasa 3: Penilaian Kepenggunaan Aplikasi & Temu bual separa berstruktur dijalankan bersama murid dan guru prasekolah untuk menilai kefahaman, minat dan kebolehgunaan aplikasi. \\
\hline
\end{tabular}
\end{table}
\section{Fasa-fasa dalam Kajian}

Kajian ini dilaksanakan mengikut tiga fasa utama berdasarkan pendekatan Reka Bentuk dan Pembangunan (DDR), iaitu fasa analisis keperluan (fasa satu), fasa reka bentuk dan pembangunan aplikasi (fasa dua), dan fasa penilaian kepenggunaan aplikasi (fasa tiga). Setiap fasa memainkan peranan penting dalam memastikan pembangunan aplikasi \textit{AR Alphabets Prasekolah} dijalankan secara sistematik dan berasaskan keperluan sebenar pengguna.

\subsection{Fasa 1: Analisis Keperluan}

Fasa pertama dalam pendekatan DDR ialah fasa analisis keperluan. Fasa ini amat penting kerana ia membantu penyelidik mengenal pasti persoalan kajian dan keperluan pengguna sasaran sebelum pembangunan aplikasi dijalankan (Ridhuan et al., 2014). Dalam konteks kajian ini, fasa ini bertujuan untuk mengenal pasti keperluan murid dan guru prasekolah terhadap aplikasi pembelajaran berasaskan realiti terimbuh (AR) bagi topik pengenalan huruf.

Menurut McKillip (1987), analisis keperluan melibatkan proses mengenal pasti dan menilai keperluan sesuatu perkara yang akan menentukan hala tuju penyelesaian. Proses ini juga dikenali sebagai proses mengenal pasti masalah dalam kalangan populasi sasaran (target population), serta mengenal pasti pendekatan terbaik untuk menyelesaikan masalah tersebut (Witkin \& Altschuld, 1995). Riviere (1996) menegaskan bahawa analisis keperluan lebih memfokuskan kepada apa yang sepatutnya berlaku (what ought to be) berbanding apa yang sedang berlaku (what is).

McKillip (1987) turut mencadangkan beberapa model yang bolehModel ini menekankan tiga komponen utama. Pertama, proses menetapkan apa yang sepatutnya dilakukan. Kedua, proses pengukuran prestasi semasa. Ketiga, proses mengenal pasti ketidaksesuaian (discrepancy identification), iaitu jurang antara apa yang sepatutnya berlaku (\textit{what ought to be}) dengan apa yang sedang berlaku (\textit{what is}).

\item \textbf{Model Pemasaran (Marketing Model)} – Model ini memberi tumpuan kepada proses menganalisis keperluan dan maklum balas pengguna bagi menilai perkara yang diperlukan dalam sesuatu perkhidmatan atau produk. Dalam konteks pendidikan, model ini sesuai digunakan untuk mengenal pasti keperluan pengguna sasaran seperti guru dan murid. Terdapat tiga komponen utama dalam model ini:

\begin{itemize}
  \item \textbf{Pemilihan populasi sasaran} – Melibatkan pemilihan kumpulan pengguna yang berpotensi tinggi untuk menggunakan produk atau perkhidmatan yang dibangunkan.
  \item \textbf{Pengkuantitian (quantification)} – Proses mengukur dan menilai keperluan pengguna serta menganalisis nilai dan minat mereka terhadap produk.
  \item \textbf{Sintesis (synthesis)} – Penyediaan indeks keperluan yang memberi gambaran menyeluruh tentang keperluan sebenar pengguna dan maklumat berkaitan produk yang dicadangkan.
\end{itemize}
Namun demikian, berdasarkan ketiga-tiga model yang dibincangkan, penyelidik memilih untuk menggunakan Model Ketidaksesuaian (Discrepancy Model) sebagai model pendukung dalam fasa analisis keperluan.

Penyelidik menggunakan teknik soal selidik dan temu bual untuk menilai sejauh mana keperluan terhadap pembangunan aplikasi \textit{AR Alphabets Prasekolah}. Pandangan guru prasekolah dan pakar pendidikan awal kanak-kanak dijadikan asas dalam mengenal pasti keperluan pengguna. Dapatan yang diperoleh daripada fasa ini digunakan sebagai asas untuk mereka bentuk aplikasi yang bersesuaian dengan konteks pembelajaran huruf bagi murid prasekolah.

Dalam aspek analisis keperluan ini, penyelidik menggariskan prosedur berikut:

\begin{itemize}
  \item Mengenal pasti kumpulan sasaran yang terdiri daripada guru dan murid prasekolah.
  \item Menjalankan soal selidik kepada 50 responden untuk mengenal pasti keperluan pengguna terhadap aplikasi pembelajaran huruf berasaskan AR.
  \item Menjalankan temu bual bersama dua orang pakar pendidikan awal kanak-kanak untuk mendapatkan pandangan profesional.
\end{itemize}

\subsubsection{Soal Selidik untuk Fasa Analisis Keperluan}

Fasa ini memfokuskan kepada keperluan murid dan guru prasekolah terhadap aplikasi pembelajaran huruf menggunakan teknologi realiti terimbuh. Seramai 50 responden terlibat dalam soal selidik ini, terdiri daripada ibu bapa dan guru yang mewakili pengguna sasaran aplikasi. Soal selidik ini direka bentuk untuk mengenal pasti tahap pendedahan terhadap teknologi, keperluan pembelajaran literasi awal, serta kesediaan menggunakan aplikasi AR dalam konteks prasekolah.

Responden yang terlibat dalam fasa ini diringkaskan dalam Jadual~\ref{jadual:respondenAnalisis}.
\begin{table}[H]
\centering
\caption{Responden Kajian Fasa Satu (Analisis Keperluan)}
\label{jadual:respondenAnalisis}
\begin{tabular}{|p{6cm}|c|}
\hline
\textbf{Responden Kajian} & \textbf{Jumlah} \\
\hline
Guru Prasekolah Zon Bangsar & 20 \\
\hline
Guru Prasekolah Zon Keramat & 15 \\
\hline
Guru Prasekolah Zon Sentul & 15 \\
\hline
\textbf{Jumlah Responden} & \textbf{50 orang} \\
\hline
\end{tabular}
\end{table}
\subsubsection{Instrumen Kajian Fasa Analisis Keperluan}

Instrumen soal selidik digunakan dalam fasa pertama kajian untuk mendapatkan maklum balas berkaitan keperluan terhadap pembangunan aplikasi \textit{AR Alphabets Prasekolah}. Soal selidik ini dibina secara berstruktur dan diubah suai berdasarkan instrumen kajian terdahulu seperti kajian reka bentuk modul m-Pembelajaran Bahasa Arab oleh Amani Dahaman (2014) dan kajian m-Pembelajaran di sekolah menengah oleh Ahmad Sobri (2010).

Soal selidik ini terdiri daripada empat bahagian utama:

\begin{itemize}
  \item \textbf{Bahagian I: Maklumat Demografi} – Mengumpul maklumat latar belakang responden seperti umur, pengalaman mengajar, dan tahap pendedahan terhadap teknologi.
  \item \textbf{Bahagian II: Penggunaan Teknologi Mudah Alih} – Mengandungi item berskala Likert lima mata: (1) Sangat Tidak Kerap, (2) Tidak Kerap, (3) Tidak Pasti, (4) Kerap, dan (5) Sangat Kerap.
  \item \textbf{Bahagian III, IV dan V: Persepsi terhadap Aplikasi AR} – Mengandungi item berskala Likert lima mata: (1) Sangat Tidak Setuju, (2) Tidak Setuju, (3) Tidak Pasti, (4) Setuju, dan (5) Sangat Setuju. Bahagian ini merangkumi aspek keperluan aplikasi, reka bentuk kandungan, dan kesesuaian teori pembelajaran.
\end{itemize}

Ringkasan pembinaan instrumen soal selidik bagi fasa analisis keperluan ditunjukkan dalam Jadual~\ref{jadual:instrumenAnalisis}.
\begin{table}[h]
\centering
\caption{Ringkasan Pembinaan Instrumen Fasa Analisis Keperluan}
\label{jadual:instrumenAnalisis}
\begin{tabular}{|p{6cm}|p{5cm}|c|}
\hline
\textbf{Item / Pembolehubah} & \textbf{Sumber Rujukan} & \textbf{Bilangan Item} \\
\hline
Penggunaan Teknologi Mudah Alih & Ahmad Sobri (2010) & 10 \\
\hline
Pengetahuan tentang Aplikasi AR & Ahmad Sobri (2010) & 12 \\
\hline
Reka Bentuk Kandungan Aplikasi & Amani Dahaman (2014) & 13 \\
\hline
Strategi Pengajaran dan Pembelajaran & Amani Dahaman (2014) & 10 \\
\hline
Aktiviti dalam Aplikasi AR & Amani Dahaman (2014) & 12 \\
\hline
Bentuk Penilaian dalam Aplikasi & Ahmad Sobri (2010) & 12 \\
\hline
\textbf{Jumlah Keseluruhan Item} & & \textbf{69} \\
\hline
\end{tabular}
\end{table}
\chapter{Kaedah Kajian}

\section{Pengenalan}

Bab ini membincangkan metodologi penyelidikan yang digunakan dalam pembangunan aplikasi \textit{AR Alphabets Prasekolah}. Metodologi merujuk kepada pendekatan sistematik yang digunakan oleh penyelidik untuk memperoleh data dan maklumat bagi mencapai objektif kajian (Mohd Majid, 1998). Kajian ini menggunakan pendekatan Kajian Reka Bentuk dan Pembangunan (Design and Development Research, DDR) yang sesuai untuk pembangunan aplikasi pendidikan interaktif. Bab ini turut menghuraikan reka bentuk kajian, prosedur pelaksanaan, kaedah persampelan, instrumen kajian, kaedah pengumpulan dan analisis data, serta langkah-langkah untuk memastikan kesahan dan kebolehpercayaan kajian.

\section{Reka Bentuk Kajian}

Kajian ini menggunakan pendekatan DDR seperti yang digariskan oleh Richey dan Klein (2007), yang merangkumi tiga fasa utama: analisis keperluan, reka bentuk dan pembangunan, serta penilaian kepenggunaan. Pendekatan ini dipilih kerana ia membolehkan pembangunan produk pendidikan yang berasaskan keperluan sebenar pengguna, serta menyediakan kerangka sistematik untuk menilai keberkesanan aplikasi dalam konteks sebenar.

\begin{itemize}
  \item \textbf{Fasa 1: Analisis Keperluan} – Mengenal pasti keperluan pengguna terhadap aplikasi AR dalam pembelajaran huruf prasekolah.
  \item \textbf{Fasa 2: Reka Bentuk dan Pembangunan} – Membangunkan aplikasi menggunakan Unity3D, Vuforia SDK dan Blender berdasarkan dapatan pakar melalui teknik Fuzzy Delphi.
  \item \textbf{Fasa 3: Penilaian Kepenggunaan} – Menilai kebolehgunaan aplikasi melalui temu bual dan pemerhatian terhadap murid prasekolah.
\end{itemize}

\section{Prosedur Kajian}

Prosedur kajian dilaksanakan mengikut ketiga-tiga fasa DDR seperti berikut:

\begin{table}[H]
\centering
\caption{Prosedur Kajian Mengikut Fasa DDR}
\label{jadual:prosedurDDR}
\begin{tabular}{|p{3cm}|p{10cm}|}
\hline
\textbf{Fasa Kajian} & \textbf{Aktiviti Utama} \\
\hline
Analisis Keperluan & Temu bual bersama guru prasekolah untuk mengenal pasti keperluan pembelajaran huruf dan potensi penggunaan AR. \\
\hline
Reka Bentuk dan Pembangunan & Pembangunan aplikasi menggunakan Unity3D, Vuforia SDK dan Blender. Dapatan pakar dianalisis menggunakan teknik Fuzzy Delphi. \\
\hline
Penilaian Kepenggunaan & Pemerhatian dan temu bual bersama murid prasekolah untuk menilai kefahaman, minat dan kebolehgunaan aplikasi. \\
\hline
\end{tabular}
\end{table}

\section{Persampelan}

Persampelan kajian ini menggunakan kaedah persampelan bertujuan (purposive sampling) kerana melibatkan kumpulan sasaran yang khusus, iaitu guru prasekolah dan murid prasekolah. Seramai lima orang guru prasekolah terlibat dalam fasa analisis keperluan, manakala lapan orang murid prasekolah dan dua orang guru terlibat dalam fasa penilaian kepenggunaan.

\section{Instrumen Kajian}

Instrumen kajian yang digunakan adalah seperti berikut:

\begin{itemize}
  \item \textbf{Temu Bual Separuh Berstruktur} – Digunakan dalam fasa analisis keperluan dan penila—
  \subsubsection{Analisis Data Fasa Analisis Keperluan}

Analisis data bagi fasa analisis keperluan melibatkan dapatan soal selidik daripada 50 orang responden yang terdiri daripada guru prasekolah di Zon Bangsar, Keramat dan Sentul. Data dianalisis menggunakan perisian \textit{Statistical Package for the Social Sciences} (SPSS) versi 21.0. Analisis deskriptif seperti kekerapan dan min digunakan untuk menentukan keperluan terhadap pembangunan aplikasi \textit{AR Alphabets Prasekolah} berdasarkan persepsi responden.

Jadual~\ref{jadual:interpretasiMin} menunjukkan interpretasi skor min yang digunakan dalam analisis ini, diadaptasi daripada kajian Amani Dahaman (2014), Gazilah Mohd Isa (2012), Ahmad Sobri Shuib (2010) dan Nik Zaharah Nik Yaacob (2007).

\begin{table}[H]
\centering
\caption{Interpretasi Skor Min Analisis Keperluan}
\label{jadual:interpretasiMin}
\begin{tabular}{|c|l|}
\hline
\textbf{Skor Min} & \textbf{Interpretasi} \\
\hline
4.01–5.00 & Tinggi \\
3.01–4.00 & Sederhana Tinggi \\
2.01–3.00 & Sederhana Rendah \\
1.00–2.00 & Rendah \\
\hline
\end{tabular}
\end{table}

\subsubsection{Kesahan dan Kebolehpercayaan Instrumen Kajian Fasa Analisis Keperluan}

Untuk memastikan kebolehpercayaan instrumen kajian, satu kajian rintis telah dijalankan bagi menilai beberapa aspek penting, termasuk:

\begin{itemize}
  \item Kefahaman terhadap kehendak soalan
  \item Tempoh masa menjawab
  \item Kesulitan lain yang mungkin dihadapi oleh responden
\end{itemize}

Sebelum kajian rintis dijalankan, penyelidik melaksanakan proses kesahan kandungan (\textit{content validity}) dengan melibatkan dua orang pakar dalam bidang pendidikan awal kanak-kanak dan teknologi pendidikan. Panel pakar ini terdiri daripada pensyarah universiti dan guru berpengalaman. Mereka menilai kesesuaian laras bahasa, struktur item, dan keselarasan konstruk dengan objektif kajian.

Hasil penilaian menunjukkan bahawa komponen dan elemen yang digunakan adalah sesuai dan menepati konteks pembelajaran prasekolah. Bahasa yang digunakan juga mudah difahami dan bersesuaian dengan tahap responden.

\subsubsection{Kajian Rintis Fasa Analisis Keperluan}

Kajian rintis dijalankan sebelum soal selidik diedarkan kepada sampel kajian sebenar. Tujuan kajian rintis adalah untuk menilai ciri-ciri psikometrik instrumen, termasuk kejelasan item, format, dan skala pengukuran. Seramai 10 orang guru prasekolah terlibat dalam kajian rintis ini.

Data daripada kajian rintis dianalisis menggunakan perisian IBM SPSS versi 21.0 untuk menentukan tahap kebolehpercayaan instrumen melalui kaedah Konsistensi Dalaman (\textit{Internal Consistency}). Menurut Chua (2006), kaedah ini sering digunakan dengan mengira pekali kebolehpercayaan Cronbach's Alpha.

Borang soal selidik ditadbir sendiri oleh penyelidik bagi memastikan kesahan dan kebolehpercayaan instrumen. Menurut Nunnally dan Bernstein (1994), pendapat pakar boleh digunakan untuk menilai kesahan kandungan. Kajian rintis ini tidak bertujuan untuk membuat generalisasi, tetapi untuk memastikan kejelasan dan kesesuaian item sebelum digunakan dalam kajian sebenar.
Menurut Churchill (1979), pendekatan ini amat sesuai digunakan dalam kajian lapangan kerana ia hanya memerlukan satu pentadbiran pengukuran bagi sesuatu instrumen. Nunnally (1978) turut menyatakan bahawa nilai kebolehpercayaan yang baik bagi sesuatu instrumen mestilah melebihi 0.70.

Dapatan kajian rintis menunjukkan bahawa semua konstruk yang diuji memperlihatkan nilai pekali kebolehpercayaan Cronbach's Alpha melebihi 0.70, menandakan tahap kebolehpercayaan yang memuaskan. Sebanyak tujuh konstruk telah diuji dalam kajian ini, iaitu:

\begin{itemize}
  \item Penggunaan teknologi mudah alih
  \item Pengetahuan tentang aplikasi AR
  \item Kepentingan aplikasi AR dalam pembelajaran
  \item Reka bentuk kandungan aplikasi
  \item Strategi pengajaran dan pembelajaran
  \item Aktiviti dalam aplikasi AR
  \item Bentuk penilaian dalam aplikasi
\end{itemize}

Ringkasan dapatan ujian rintis ditunjukkan dalam Jadual~\ref{jadual:cronbachAlpha}.
\begin{table}[H]
\centering
\caption{Ringkasan Dapatan Ujian Rintis (Nilai Cronbach's Alpha)}
\label{jadual:cronbachAlpha}
\begin{tabular}{|p{8cm}|c|}
\hline
\textbf{Konstruk} & \textbf{Nilai Cronbach's Alpha} \\
\hline
Penggunaan Teknologi Mudah Alih & 0.82 \\
Pengetahuan tentang Aplikasi AR & 0.85 \\
Kepentingan Aplikasi AR dalam Pembelajaran & 0.81 \\
Reka Bentuk Kandungan Aplikasi & 0.87 \\
Strategi Pengajaran dan Pembelajaran & 0.80 \\
Aktiviti dalam Aplikasi AR & 0.83 \\
Bentuk Penilaian dalam Aplikasi & 0.84 \\
\hline
\textbf{Keseluruhan Instrumen} & \textbf{0.85} \\
\hline
\end{tabular}
\end{table}
Secara keseluruhannya, proses analisis keperluan dalam kajian ini telah dilaksanakan secara sistematik melalui pembinaan instrumen yang sah dan boleh dipercayai. Dapatan kajian rintis menunjukkan bahawa semua konstruk yang diuji mencapai tahap kebolehpercayaan yang memuaskan, justeru instrumen ini sesuai digunakan dalam kajian sebenar untuk mengenal pasti keperluan pengguna terhadap pembangunan aplikasi \textit{AR Alphabets Prasekolah}.
\subsection{Analisis Temu Bual Pakar untuk Fasa Analisis Keperluan}

Fasa analisis keperluan turut menggunakan kaedah temu bual berbentuk soalan terbuka (\textit{open-ended questions}) bagi mendapatkan pandangan pakar tentang keperluan reka bentuk aplikasi \textit{AR Alphabets Prasekolah}. Temu bual ini dijalankan secara bertulis melalui borang soal selidik terbuka yang diedarkan kepada dua orang pakar dalam bidang pendidikan awal kanak-kanak dan teknologi pendidikan. Maklumat demografi responden pakar ditunjukkan dalam Jadual~\ref{jadual:demografiPakar}.

\begin{table}[H]
\centering
\caption{Demografi Responden Pakar}
\label{jadual:demografiPakar}
\begin{tabular}{|c|c|c|c|p{5cm}|}
\hline
\textbf{Responden} & \textbf{Jantina} & \textbf{Kaum} & \textbf{Agama} & \textbf{Bidang Kerja} \\
\hline
P1 & Perempuan & Melayu & Islam & Guru Cemerlang Prasekolah \\
\hline
P2 & Lelaki & Melayu & Islam & Guru Pakar Pendidikan Awal Kanak-kanak \\
\hline
\end{tabular}
\end{table}

Menurut Othman (2007), kaedah kualitatif bukan sahaja digunakan untuk mengenal pasti elemen baharu seperti dalam kajian sains semula jadi, tetapi juga untuk mengukuhkan kefahaman terhadap sesuatu isu melalui dialog dan refleksi. Kaedah ini membolehkan penyelidik memperoleh perspektif yang lebih mendalam dan pelbagai terhadap fenomena yang dikaji. Strauss dan Corbin (1998) turut menyatakan bahawa pendekatan kualitatif sesuai digunakan untuk memahami fenomena yang belum diterokai secara menyeluruh.

Berdasarkan pendekatan ini, penyelidik menggunakan soalan terbuka untuk mendapatkan pandangan pakar terhadap keperluan pembangunan aplikasi AR dalam konteks pembelajaran huruf prasekolah. Soalan-soalan tersebut disusun mengikut tiga subtema utama:

\begin{itemize}
  \item Pandangan pakar terhadap keperluan reka bentuk aplikasi AR untuk pembelajaran huruf di prasekolah
  \item Pandangan terhadap konstruk kajian yang dicadangkan
  \item Pandangan pakar terhadap peluang dan cabaran pelaksanaan pengajaran dan pembelajaran menggunakan teknologi AR dalam konteks prasekolah
\end{itemize}
Instrumen kajian bagi fasa analisis keperluan turut merangkumi temu bual berbentuk soalan terbuka. Soalan-soalan ini dirangka berdasarkan komponen utama yang akan dimasukkan ke dalam aplikasi \textit{AR Alphabets Prasekolah}. Fokus utama adalah untuk mendapatkan pandangan pakar tentang keperluan reka bentuk dan pembangunan aplikasi pembelajaran huruf berasaskan teknologi realiti terimbuh (AR) dalam konteks pendidikan prasekolah.

Berikut merupakan senarai soalan temu bual yang dikemukakan kepada pakar:

\begin{enumerate}
  \item Pada pandangan anda, adakah terdapat keperluan untuk mereka bentuk dan membangunkan aplikasi pembelajaran huruf berasaskan AR untuk murid prasekolah? Sila jelaskan.
  
  \item Apakah jenis perkakasan teknologi mudah alih yang sesuai digunakan dalam pembangunan aplikasi AR untuk pembelajaran huruf di prasekolah? Sila jelaskan.
  
  \item Apakah jenis perisian atau platform yang sesuai digunakan dalam pembangunan aplikasi AR untuk pembelajaran huruf? Sila jelaskan.
  
  \item Pada pandangan anda, apakah aktiviti pengajaran dan pembelajaran yang sesuai dimasukkan ke dalam aplikasi AR ini? Sila jelaskan.
  
  \item Apakah strategi pengajaran yang sesuai untuk diaplikasikan dalam penggunaan aplikasi AR di peringkat prasekolah? Sila jelaskan.
  
  \item Apakah bentuk penilaian yang sesuai digunakan dalam aplikasi AR untuk menilai kefahaman dan penglibatan murid? Sila jelaskan.
  
  \item Pada pandangan anda, mengapakah teknologi AR perlu diterapkan dalam pengajaran dan pembelajaran di peringkat prasekolah?
\end{enumerate}

Fasa satu kajian ini telah dijalankan secara sistematik dan mengikut urutan seperti yang diringkaskan dalam Rajah~\ref{rajah:fasa1} di halaman~\pageref{rajah:fasa1}.



\section{Fasa Reka Bentuk dan Pembangunan Aplikasi}

Fokus utama fasa ini adalah untuk mereka bentuk dan membangunkan aplikasi \textit{AR Alphabets Prasekolah} berdasarkan keperluan pengguna yang telah dikenalpasti dalam fasa sebelumnya. Fasa ini merupakan komponen terpenting dalam kajian kerana ia melibatkan proses pembentukan struktur aplikasi, pemilihan kandungan, dan pemetaan elemen pembelajaran yang sesuai untuk murid prasekolah.

Perbincangan dalam fasa ini merangkumi aspek populasi pakar, instrumen kajian, proses analisis data, serta pembentukan reka bentuk aplikasi. Teknik \textit{Fuzzy Delphi Method} (FDM) digunakan sebagai pendekatan utama dalam fasa ini. FDM merupakan gabungan antara teori set kabur (Fuzzy Set Theory) dan Teknik Delphi tradisional. Teknik ini diperkenalkan oleh Murray, Pipino dan Gigch (1985) dan dikembangkan oleh Kaufman dan Gupta (1988). Ia bertujuan untuk meningkatkan ketepatan dan kesepakatan dalam kalangan pakar melalui proses penilaian berulang yang lebih sistematik (Mohd Ridhuan et al., 2013).
\subsection{Sampel Kajian Fasa Reka Bentuk dan Pembangunan}

Dalam fasa ini, penyelidik memilih dua orang pakar sebagai panel penilai menggunakan pendekatan \textit{Fuzzy Delphi}. Pemilihan pakar dibuat secara bertujuan berdasarkan kepakaran dalam bidang pendidikan awal kanak-kanak dan teknologi pendidikan. Kriteria pemilihan pakar adalah seperti berikut:

\begin{itemize}
  \item Huru Cemerlanag atau guru pakar dalam bidang pendidikan awal kanak-kanak
  \item Mempunyai pengalaman lebih daripada lima tahun dalam pembangunan bahan bantu mengajar
  \item Terlibat secara aktif dalam penyelidikan berkaitan teknologi pendidikan atau aplikasi AR
  \item Mempunyai kelayakan akademik sekurang-kurangnya di peringkat Sarjana atau Doktor Falsafah
\end{itemize}

Menurut Berliner (2004), individu yang telah berkhidmat antara lima hingga sepuluh tahun dalam bidang pendidikan boleh dikategorikan sebagai pakar kerana mereka telah menjalani pengalaman pengajaran dan pengurusan secara berterusan. Dalam konteks Teknik Fuzzy Delphi, pemilihan pakar adalah aspek paling kritikal. Dalkey (1972) menyatakan bahawa pakar dalam kajian Delphi ialah individu yang mempunyai pengetahuan dan kemahiran mendalam dalam bidang tertentu.

Walaupun jumlah ideal pakar dalam kajian Fuzzy Delphi adalah antara 10 hingga 50 orang (Jones \& Twiss, 1978), kajian ini menggunakan dua orang pakar berdasarkan skop dan skala kajian yang bersifat eksploratori. Adler dan Ziglio (1996) turut menyatakan bahawa jumlah pakar serendah 10 hingga 15 orang masih memadai sekiranya tahap kesepakatan adalah tinggi. Oleh itu, pemilihan dua pakar dalam kajian ini adalah wajar dan mencukupi untuk mendapatkan pandangan pakar secara mendalam dan fokus.
\subsection{Instrumen Kajian Fasa Reka Bentuk dan Pembangunan}

Fasa kedua kajian ini merupakan peringkat mereka bentuk aplikasi \textit{AR Alphabets Prasekolah}. Ia dilaksanakan menggunakan Teknik \textit{Fuzzy Delphi} bagi mendapatkan kesepakatan pakar terhadap elemen-elemen penting yang perlu dimasukkan ke dalam aplikasi. Teknik ini dipilih kerana ia merupakan kaedah yang sistematik dan berkesan untuk memperoleh konsensus dalam kalangan pakar melalui penggunaan formula matematik.

Instrumen kajian yang digunakan dalam fasa ini ialah borang soal selidik yang dibina berdasarkan kajian lepas dan diubah suai mengikut keperluan kajian ini. Soal selidik ini mengandungi item-item yang mengukur konstruk seperti isi kandungan aplikasi, objektif pembelajaran, jenis aplikasi, perkakasan, strategi pengajaran, dan bentuk penilaian.

Instrumen ini dibina dengan mengadaptasi dan menggabungkan item daripada kajian Amani Dahaman (2014), Ahmad Sobri (2010), dan Ibrahem Narongraksakhet (2003). Ringkasan pembinaan instrumen ditunjukkan dalam Jadual~\ref{jadual:instrumenFDM}.
\begin{table}[H]
\centering
\caption{Ringkasan Instrumen Kajian Fasa Reka Bentuk dan Pembangunan}
\label{jadual:instrumenFDM}
\begin{tabular}{|p{5cm}|p{4cm}|p{4cm}|c|}
\hline
\textbf{Item / Pembolehubah} & \textbf{Sumber Rujukan} & \textbf{Sumber Asal} & \textbf{Bilangan Item} \\
\hline
Isi Kandungan Aplikasi & Amani Dahaman (2014) & Ahmad Sobri (2010) & 7 \\
\hline
Objektif Pembelajaran & Amani Dahaman (2014) & Ahmad Sobri (2010) & 9 \\
\hline
Jenis Aplikasi & Ahmad Sobri (2010) & Ibrahem Narongraksakhet (2003) & 9 \\
\hline
Jenis Perkakasan & Ahmad Sobri (2010) & Ibrahem Narongraksakhet (2003) & 5 \\
\hline
Strategi Pengajaran & Ahmad Sobri (2010) & Ibrahem Narongraksakhet (2003) & 3 \\
\hline
Bentuk Penilaian & Amani Dahaman (2014) & Ahmad Sobri (2010) & 8 \\
\hline
\textbf{Jumlah Keseluruhan Item} & & & \textbf{41} \\
\hline
\end{tabular}
\end{table}
Pengukuran bagi setiap konstruk telah digabungkan untuk membentuk satu instrumen penyelidikan yang lengkap. Instrumen ini disusun mengikut lapan bahagian utama seperti berikut:

\begin{itemize}
  \item \textbf{Bahagian A:} Maklumat Demografi Responden
  \item \textbf{Bahagian B:} Objektif Aplikasi \textit{AR Alphabets Prasekolah}
  \item \textbf{Bahagian C:} Isi Kandungan Aplikasi \textit{AR Alphabets Prasekolah}
  \item \textbf{Bahagian D:} Pemilihan Jenis Aplikasi dan Platform Teknologi
  \item \textbf{Bahagian E:} Pemilihan Perkakasan Teknologi yang Sesuai
  \item \textbf{Bahagian F:} Strategi Pengajaran dan Pembelajaran yang Disokong oleh Aplikasi
  \item \textbf{Bahagian G:} Bentuk Penilaian dalam Aplikasi \textit{AR Alphabets Prasekolah}
  \item \textbf{Bahagian H:} Peluang dan Cabaran Pelaksanaan Aplikasi dalam Konteks Prasekolah
\end{itemize}
Antara pakar yang dicadangkan dalam kajian ini ialah Pensyarah Institut Pendidikan Guru yang mempunyai kepakaran dalam bidang Pengajian Profesional, Teknologi Maklumat dan Komunikasi (ICT), serta reka bentuk kurikulum. Pakar dihubungi melalui pelbagai saluran seperti telefon, e-mel, atau pertemuan secara bersemuka. Walaupun kajian ini melibatkan seorang  orang pakar sebagai panel utama, perbincangan metodologi tetap merujuk kepada amalan standard yang melibatkan sehingga 20 orang pakar sebagai rujukan literatur.

\subsection{Langkah-langkah dalam Menjalankan Kajian Menggunakan Kaedah Fuzzy Delphi (FDM)}

Kajian ini menggunakan data kuantitatif yang dianalisis melalui pendekatan \textit{Fuzzy Delphi Method} (FDM). Menurut Lay Yoon Fah dan Khoo Chwee Hoon (2008), pemilihan ujian statistik yang sesuai perlu mengambil kira jenis skala data, bilangan sampel, jenis ukuran, dan sama ada data bersifat parametrik atau bukan parametrik.

Bagi memastikan dapatan kajian yang diperoleh adalah sah dan empirikal, prosedur pelaksanaan FDM perlu dipatuhi secara sistematik. Langkah-langkah pelaksanaan kajian menggunakan kaedah FDM adalah seperti berikut:

\begin{itemize}
  \item \textbf{Langkah 1: Pembentukan Soalan Soal Selidik FDM} \\
  \end{itemize}
  Soalan-soalan dibina berdasarkan:
  \begin{enumerate}[label*=\roman*.]
    \item Sorotan literatur
    \item Temu bual pakar
\subsection{Langkah-langkah dalam Menjalankan Kajian Menggunakan Kaedah Fuzzy Delphi (FDM)}

Kajian ini menggunakan data kuantitatif yang dianalisis melalui pendekatan \textit{Fuzzy Delphi Method} (FDM). Menurut Lay Yoon Fah dan Khoo Chwee Hoon (2008), pemilihan ujian statistik yang sesuai perlu mengambil kira jenis skala data, bilangan sampel, jenis ukuran, serta sama ada data bersifat parametrik atau bukan parametrik.

Bagi memastikan dapatan yang diperoleh adalah sah dan empirikal, prosedur pelaksanaan FDM perlu dipatuhi secara sistematik. Langkah-langkah pelaksanaan kajian menggunakan kaedah FDM adalah seperti berikut:
\end{enumerate}
\begin{itemize}
  \item \textbf{Langkah 1: Pembentukan Soalan Soal Selidik FDM} \\
  Soalan dibina berdasarkan:
  \begin{enumerate}[label*=\roman*.]
    \item Sorotan literatur
    \item Temu bual pakar
    \item Perbincangan kumpulan fokus (jika ada)
  \end{enumerate}
  Soalan menggunakan skala Likert tujuh mata yang mengukur aras persetujuan, tahap, dan kepentingan terhadap konstruk yang dikaji.

  \item \textbf{Langkah 2: Pengumpulan Data daripada Pakar} \\
  Pakar dijemput untuk menilai kepentingan setiap pembolehubah menggunakan pembolehubah linguistik. Kaedah pengumpulan data termasuk:
  \begin{enumerate}[label*=\roman*.]
    \item Menjalankan seminar atau bengkel ilmiah
    \item Temu bual bersemuka
    \item Penyebaran soal selidik secara dalam talian (e-mel atau borang digital)
  \end{enumerate}

  \item \textbf{Langkah 3: Penukaran Pembolehubah Linguistik kepada Nombor Fuzzy} \\
  Semua pembolehubah linguistik ditukar kepada nombor segitiga fuzzy (\textit{triangular fuzzy numbers}) berdasarkan skala tujuh poin. Penukaran ini membolehkan analisis matematik dijalankan untuk menentukan tahap kesepakatan pakar. Skala fuzzy yang digunakan ditunjukkan dalam Jadual~\ref{jadual:skalaFuzzy}.
\end{itemize}

\begin{table}[H]
\centering
\caption{Aras Persetujuan dan Skala Fuzzy bagi 7 Poin}
\label{jadual:skalaFuzzy}
\begin{tabular}{|p{6cm}|c|}
\hline
\textbf{Pembolehubah Linguistik} & \textbf{Skala Fuzzy (Triangular)} \\
\hline
Sangat-sangat Tidak Setuju & (0.0, 0.0, 0.1) \\
Sangat Tidak Setuju & (0.0, 0.1, 0.3) \\
Tidak Setuju & (0.1, 0.3, 0.5) \\
Tidak Pasti & (0.3, 0.5, 0.7) \\
Setuju & (0.5, 0.7, 0.9) \\
Sangat Setuju & (0.7, 0.9, 1.0) \\
Sangat-sangat Setuju & (0.9, 1.0, 1.0) \\
\hline
\end{tabular}
\end{table}
\begin{table}[h]
\centering
\caption{Aras Persetujuan dan Skala Fuzzy bagi 5 Poin}
\label{jadual:fuzzy5poin}
\begin{tabular}{|p{6cm}|c|}
\hline
\textbf{Pembolehubah Linguistik} & \textbf{Skala Fuzzy (Triangular)} \\
\hline
Sangat Tidak Setuju & (0.0, 0.0, 0.2) \\
Tidak Setuju & (0.0, 0.2, 0.4) \\
Tidak Pasti & (0.2, 0.4, 0.6) \\
Setuju & (0.4, 0.6, 0.8) \\
Sangat Setuju & (0.6, 0.8, 1.0) \\
\hline
\end{tabular}
\end{table}

\begin{table}[h]
\centering
\caption{Aras Tahap dan Skala Fuzzy bagi 7 Poin}
\label{jadual:fuzzyTahap7}
\begin{tabular}{|p{6cm}|c|}
\hline
\textbf{Pembolehubah Linguistik} & \textbf{Skala Fuzzy (Triangular)} \\
\hline
Sangat Rendah & (0.0, 0.0, 0.1) \\
Sangat Sederhana Rendah & (0.0, 0.1, 0.3) \\
Rendah & (0.1, 0.3, 0.5) \\
Tidak Pasti & (0.3, 0.5, 0.7) \\
Tinggi & (0.5, 0.7, 0.9) \\
Sangat Sederhana Tinggi & (0.7, 0.9, 1.0) \\
Sangat Tinggi & (0.9, 1.0, 1.0) \\
\hline
\end{tabular}
\end{table}
\begin{table}[H]
\centering
\caption{Aras Kepentingan dan Skala Fuzzy bagi 5 Poin}
\label{jadual:fuzzyKepentingan5}
\begin{tabular}{|p{6cm}|c|}
\hline
\textbf{Pembolehubah Linguistik} & \textbf{Skala Fuzzy (Triangular)} \\
\hline
Sangat Penting & (0.0, 0.0, 0.2) \\
Penting & (0.0, 0.2, 0.4) \\
Sederhana Penting & (0.2, 0.4, 0.6) \\
Tidak Penting & (0.4, 0.6, 0.8) \\
Sangat Tidak Penting & (0.6, 0.8, 1.0) \\
\hline
\end{tabular}
\end{table}
\begin{itemize}
  \item \textbf{Langkah 4: Penukaran Skala Likert kepada Skala Fuzzy dan Pemprosesan Data} \\
  Setelah semua data dan maklumat diperoleh daripada pakar, pengkaji menukarkan skala Likert kepada skala fuzzy berdasarkan jadual-jadual yang telah ditetapkan. Proses ini membolehkan data dianalisis secara kuantitatif menggunakan pendekatan matematik fuzzy.

  Data yang telah ditukar kepada nombor fuzzy akan dianalisis menggunakan perisian Microsoft Excel. Analisis ini melibatkan pengiraan nilai min fuzzy (\textit{fuzzy average}), nilai threshold (\(d\)-value), dan tahap konsensus pakar.
\end{itemize}

\begin{itemize}
  \item \textbf{Langkah 5: Pengiraan Jarak dan Penentuan Konsensus} \\
  Pengiraan jarak antara purata nilai fuzzy pakar menggunakan kaedah \textit{vertex method} seperti yang dicadangkan oleh Chen (2000). Jarak antara dua nombor fuzzy \( m = (m_1, m_2, m_3) \) dan \( n = (n_1, n_2, n_3) \) dikira menggunakan rumus berikut:

  

\[
  d(m,n) = \sqrt{\frac{1}{3} \left[ (m_1 - n_1)^2 + (m_2 - n_2)^2 + (m_3 - n_3)^2 \right]}
  \]



  Menurut Cheng dan Lin (2002), jika nilai jarak (\(d\)) antara purata dengan penilaian pakar adalah kurang daripada 0.2, maka dianggap telah mencapai konsensus. Tambahan pula, jika sekurang-kurangnya 75\% daripada pakar mencapai konsensus (Chu \& Hwang, 2008; Murry \& Hammons, 1995), maka kajian boleh diteruskan ke langkah seterusnya. Jika tidak, pusingan kedua FDM perlu dijalankan atau item yang tidak mencapai konsensus boleh digugurkan.

  \item \textbf{Langkah 6: Proses Defuzzification dan Penentuan Keutamaan} \\
  Bagi setiap alternatif, nilai fuzzy \( A_i = (m_1, m_2, m_3) \) akan melalui proses \textit{defuzzification} menggunakan formula berikut:

  

\[
  A = \frac{1}{3}(m_1 + m_2 + m_3)
  \]



  Nilai defuzzified ini digunakan untuk menentukan turutan keutamaan (ranking) bagi setiap item atau elemen yang dinilai. Item dengan nilai tertinggi dianggap paling penting atau paling disepakati oleh pakar.
\end{itemize}
\subsection{Kesahan dan Kebolehpercayaan Fasa Reka Bentuk dan Pembangunan}

Kesahan alat pengukuran merujuk kepada ketepatan dan kesesuaian sesuatu instrumen dalam mengukur apa yang sepatutnya diukur (Chua, 2006; Pallant, 2001; Wiersma, 2000). Instrumen yang sah dan mempunyai kesahan pengukuran yang tinggi menjadi asas penting dalam menghasilkan dapatan kajian yang tepat, konsisten dan boleh digeneralisasikan (Yong-Mi Kim, 2009).

Dalam konteks kajian ini, kesahan merangkumi dua aspek utama:
\begin{enumerate}
  \item Kesahan instrumen soal selidik Fuzzy Delphi
  \item Kesahan prototaip aplikasi \textit{AR Alphabets Prasekolah}
\end{enumerate}

Kesahan instrumen soal selidik FDM merujuk kepada ketepatan kandungan item dalam mengukur konstruk yang ditetapkan. Instrumen ini dibina berdasarkan kajian literatur terdahulu dan disemak oleh pakar bidang melalui kaedah \textit{expert judgement}. Menurut Jusoh (2008), salah satu kriteria penting dalam kesahan kandungan ialah pemilihan item berdasarkan kajian lepas yang relevan dan sahih. Dalam kajian ini, semua item telah disemak dan ditambah baik berdasarkan komen dan cadangan pakar, termasuk dari segi struktur ayat, istilah teknikal, dan kesesuaian konteks.

Kesahan kandungan juga merupakan prosedur penting yang perlu dilaksanakan sebelum instrumen digunakan dalam kajian sebenar. Menurut Cresswell (2008), Chua (2006), dan Pallant (2001), kesahan kandungan merujuk kepada sejauh mana item dalam instrumen mencerminkan keseluruhan domain yang dikaji. Cresswell (2007) mencadangkan agar penyelidik mendapatkan pengesahan daripada pakar bidang untuk memastikan kesesuaian dan ketepatan setiap item. Dalam kajian ini, sekurang-kurangnya tiga orang pakar telah dirujuk bagi menilai kesahan kandungan instrumen, selaras dengan saranan Makki, Khalick dan BouJoude (2003).

Kesahan bahasa dan kandungan dalam kajian ini dikategorikan sebagai pendekatan kesahan rasional, di mana penilaian dibuat secara sistematik berdasarkan kepakaran dan pengalaman pakar dalam bidang pendidikan awal kanak-kanak dan teknologi pendidikan.
Kesahan yang dijalankan dalam kajian ini adalah bertujuan untuk menilai keperluan secara rasional terhadap instrumen yang dibina. Pendekatan ini selaras dengan pandangan Norlia (2010) yang menyatakan bahawa kesahan rasional melibatkan beberapa faktor penting dalam pembinaan item, iaitu:

\begin{enumerate}
  \item Item dibina berdasarkan pemikiran, kepercayaan, dan disahkan oleh pakar dalam bidang berkaitan;
  \item Item dirangka berdasarkan teori yang merujuk kepada jangkaan tingkah laku yang dapat mentafsirkan teori tersebut;
  \item Item yang dibina adalah berkesan dan mempunyai kesahan yang tinggi.
\end{enumerate}

Murphy dan Davidshofer (1998) turut menegaskan bahawa pendekatan kesahan empirikal hanya diperlukan dalam kajian yang melibatkan penilaian psikometrik. Oleh itu, dalam konteks kajian ini, penggunaan pendekatan kesahan rasional adalah memadai dan sesuai.

Penyelidik telah melantik empat orang pakar dari Universiti Malaya, Universiti Utara Malaysia, dan Institut Pendidikan Guru untuk menilai kesahan instrumen kajian. Panel pakar ini terlibat dalam proses pengesahan instrumen soal selidik bagi fasa analisis keperluan dan juga dalam pelaksanaan Teknik Fuzzy Delphi untuk fasa reka bentuk dan pembangunan.

Hasil daripada proses pengesahan menunjukkan bahawa pakar bersetuju dengan pemilihan komponen dan elemen yang digunakan dalam kajian. Bahasa yang digunakan dalam instrumen juga didapati sesuai dan mudah difahami. Komponen utama yang digunakan dalam pembangunan aplikasi \textit{AR Alphabets Prasekolah} menepati kriteria yang diperlukan, dan pendekatan yang digunakan dalam reka bentuk aplikasi dilihat selari dengan
Kesahan yang dijalankan dalam kajian ini adalah bertujuan untuk menilai keperluan secara rasional terhadap instrumen yang dibina. Pendekatan ini selaras dengan pandangan Norlia (2010) yang menyatakan bahawa kesahan rasional melibatkan beberapa faktor penting dalam pembinaan item, iaitu:

\begin{enumerate}
  \item Item dibina berdasarkan pemikiran, kepercayaan, dan disahkan oleh pakar dalam bidang berkaitan;
  \item Item dirangka berdasarkan teori yang merujuk kepada jangkaan tingkah laku yang dapat mentafsirkan teori tersebut;
  \item Item yang dibina adalah berkesan dan mempunyai kesahan yang tinggi.
\end{enumerate}

Murphy dan Davidshofer (1998) turut menegaskan bahawa pendekatan kesahan empirikal hanya diperlukan dalam kajian yang melibatkan penilaian psikometrik. Oleh itu, dalam konteks kajian ini, penggunaan pendekatan kesahan rasional adalah memadai dan sesuai.

Penyelidik telah melantik empat orang pakar dari Universiti Malaya, Universiti Utara Malaysia, dan Institut Pendidikan Guru untuk menilai kesahan instrumen kajian. Panel pakar ini terlibat dalam proses pengesahan instrumen soal selidik bagi fasa analisis keperluan dan juga dalam pelaksanaan Teknik Fuzzy Delphi untuk fasa reka bentuk dan pembangunan.

Hasil daripada proses pengesahan menunjukkan bahawa pakar bersetuju dengan pemilihan komponen dan elemen yang digunakan dalam kajian. Bahasa yang digunakan dalam instrumen juga didapati sesuai dan mudah difahami. Komponen utama yang digunakan dalam pembangunan aplikasi \textit{AR Alphabets Prasekolah} menepati kriteria yang diperlukan, dan pendekatan yang digunakan dalam reka bentuk aplikasi dilihat selari dengan keperluan pengajaran dan pembelajaran di peringkat prasekolah.

Fasa kedua kajian ini telah dijalankan secara sistematik mengikut urutan seperti yang diringkaskan dalam Rajah~\ref{rajah:fasa2} di halaman~\pageref{rajah:fasa2}.
\section{Fasa Penilaian Kepenggunaan}

Fasa ini merujuk kepada penilaian akhir dalam kajian reka bentuk dan pembangunan aplikasi \textit{AR Alphabets Prasekolah}. Tujuan utama fasa ini adalah untuk menilai tahap kepenggunaan (\textit{usability}) aplikasi yang telah dibangunkan berdasarkan maklum balas pengguna sasaran. Penilaian ini penting bagi memastikan aplikasi yang dihasilkan benar-benar memenuhi keperluan pengguna dari aspek kefungsian, kemudahan penggunaan, dan keberkesanan dalam konteks pembelajaran huruf di prasekolah.
\subsection{Sampel Kajian Fasa Penilaian Kepenggunaan}

Dalam fasa ketiga ini, seramai diua  guru prasekolah dan seorang orang pensyarah pendidikan awal kanak-kanak dari IPG  Teknik terlibat sebagai responden. Mereka dipilih berdasarkan pengalaman dalam pengajaran literasi awal dan penggunaan teknologi pendidikan. Sampel ini dipilih secara bertujuan untuk mendapatkan maklum balas yang mendalam dan relevan terhadap aplikasi yang dibangunkan.
\subsection{Instrumen Kajian Fasa Penilaian Kepenggunaan}

 Penilaian Kebolehgunaan: System Usability Scale (SUS)

System Usability Scale (SUS) merupakan satu instrumen penilaian kebolehgunaan yang dibangunkan oleh John Brooke pada tahun 1986. Ia direka bentuk untuk menilai persepsi pengguna terhadap kemudahan penggunaan sesuatu sistem secara subjektif. SUS terdiri daripada \textbf{10 item soal selidik} yang menggunakan \textbf{skala Likert lima mata}, merangkumi aspek-aspek seperti kefahaman, konsistensi, dan keyakinan pengguna terhadap sistem yang diuji.

Skor keseluruhan SUS dikira berdasarkan formula standard dan menghasilkan nilai antara \textbf{0 hingga 100}, di mana skor yang lebih tinggi menunjukkan tahap kebolehgunaan yang lebih baik. Instrumen ini telah digunakan secara meluas dalam pelbagai bidang, termasuk teknologi pendidikan, kerana sifatnya yang ringkas, pantas, dan boleh dipercayai.


Manakla Instrumen temu bual  yang digunakan  disemak olehseorang orang panel pakar yang turut terlibat dalam penyemakan instrumen kajian sebelumnya. Proses kesahan melibatkan kesahan dalaman (\textit{internal validity}) yang merangkumi kesahan kriteria (\textit{criteria validity}) dan semakan rentas (\textit{cross-checking}) terhadap item-item soalan.

Temu bual ini mengandungi 23 item yang merangkumi dua peringkat utama:
\item 
\begin{enumerate}
  \item \textbf{Peringkat Pertama:} Soalan berkaitan demografi dan kemahiran asas responden dalam penggunaan teknologi.
  \item \textbf{Peringkat Kedua:} Soalan berkaitan pedagogi kesan pengguna aplikasi keatas murid  dan  penilaian kebolehgunaan
\end{enumerate}
\subsection{Prosedur Pengumpulan Data Fasa Penilaian Kepenggunaan}

Prosedur pengumpulan data dalam fasa penilaian kepenggunaan dibahagikan kepada lima peringkat utama seperti berikut:

\begin{enumerate}
  \item \textbf{Peringkat Pertama: Perancangan Pelaksanaan Kajian} \\
  Pada peringkat ini, pengkaji menjalankan proses pemilihan lokasi kajian, pemilihan responden, dan pembinaan soalan temu bual separa struktur. Prasekolah dipilih sebagai lokasi kajian kerana kemudahan rangkaian tanpa wayar (wireless) yang stabil dan meliputi hampir 90\% kawasan persekitaran sekolah. Soalan temu bual yang dibina disemak oleh panel pakar bagi memastikan kesahan kandungan. Seterusnya, satu sesi taklimat kajian diberikan kepada responden sebelum pelaksanaan sesi pengajaran dan temu bual dijalankan.

  \item \textbf{Peringkat Kedua: Pelaksanaan Kajian – Ujian Alfa} \\
  Kajian ini dilaksanakan dalam tempoh satu minggu dan melibatkan Guru Pakar Prasekolah dalam membantu penguji menguji prototaip yang dibangunkan.

  \item \textbf{Peringkat Ketiga: Analisis Data Ujian Alfa} \\
  Pada peringkat ini, pengkaji menganalisis dokumen borang ujian kefungsian. Analisis dijalankan secara tematik berdasarkan aspek-aspek utama dalam objektif dan persoalan kajian, termasuk aspek teknologi, kefungsian, kemudahan penggunaan, dan keberkesanan aplikasi dalam konteks pembelajaran huruf di prasekolah.

  \item \textbf{Peringkat Keempat: Pelaksanaan Kajian dan Pengumpulan Data SUS} \\
  Kajian ini dilaksanakan dalam tempoh dua bulan dan melibatkan empat sesi pembelajaran menggunakan rancangan pengajaran yang dibina berdasarkan Model ASSURE. Pelaksanaan pengajaran menggunakan aplikasi \textit{AR Alphabets Prasekolah} dijalankan dalam persekitaran sebenar bilik darjah. Responden menggunakan pelbagai perkakasan seperti telefon pintar, tablet, dan komputer riba untuk mengakses aplikasi melalui sambungan internet. Temu bual dijalankan selepas sesi pengajaran bagi mendapatkan maklum balas berkaitan aspek kepenggunaan aplikasi.

  \item \textbf{Peringkat Kelima: Analisis Transkrip dan Dokumen} \\
  Pada peringkat ini, pengkaji menganalisis transkrip temu bual dan dokumen berkaitan. Analisis dijalankan secara tematik berdasarkan aspek-aspek utama dalam objektif dan persoalan kajian, termasuk aspek teknologi, kefungsian, kemudahan penggunaan, dan keberkesanan aplikasi dalam konteks pembelajaran huruf di prasekolah.
\end{enumerate}

Analisis transkrip temu bual dibahagikan kepada dua tema utama, iaitu aspek kepenggunaan model dan aspek pedagogi. Peringkat terakhir dalam prosedur ini ialah penulisan laporan berdasarkan dapatan temu bual.

Proses lengkap pengumpulan data dalam fasa penilaian kepenggunaan diringkaskan dalam Jadual~\ref{jadual:prosesPenilaian}.

\begin{table}[H]
\centering
\caption{Proses Pengumpulan Data Fasa Penilaian Kepenggunaan}
\label{jadual:prosesPenilaianLengkap}
\begin{tabular}{|p{3.5cm}|p{3.5cm}|p{3.5cm}|p{3.5cm}|p{3.5cm}|p{3.5cm}|}
\hline
\textbf{Perkara} & \textbf{Peringkat 1} \\ \textbf{Perancangan Kajian} & \textbf{Peringkat 2} \\ \textbf{Ujian Alfa} & \textbf{Peringkat 3} \\ \textbf{Analisis Alfa} & \textbf{Peringkat 4} \\ \textbf{Pelaksanaan SUS} & \textbf{Peringkat 5} \\ \textbf{Analisis Temu Bual} \\
\hline
Pengumpulan Data & Pemilihan lokasi, responden, pembinaan dan semakan soalan temu bual & Pelaksanaan sesi pengajaran dan pengujian prototaip bersama guru pakar & Analisis borang kefungsian dan dokumen ujian & Pelaksanaan sesi pengajaran sebenar, pengumpulan data SUS, temu bual pengguna & Transkripsi dan analisis data temu bual serta dokumen berkaitan \\
\hline
Fokus Analisis & — & Pemerhatian awal dan kefungsian sistem & Aspek teknologi dan kefungsian aplikasi & Aspek kepenggunaan dan keberkesanan pembelajaran & Aspek kepenggunaan model dan pedagogi \\
\hline
Tempoh Pelaksanaan & Sebelum kajian bermula & 1 minggu & Selepas ujian alfa & 2 bulan & Selepas sesi pengajaran \\
\hline
Instrumen Digunakan & Soalan temu bual separa struktur & Borang pemerhatian & Borang kefungsian & Skala SUS, temu bual pengguna & Transkrip temu bual, dokumen refleksi \\
\hline
\end{tabular}
\end{table}

Analisis Data Fasa Penilaian Kepenggunaan}

Fasa ketiga kajian ini menumpukan kepada penilaian kepenggunaan aplikasi \textit{AR Alphabets Prasekolah}  Data kualitatif diperoleh melalui temu bual separa struktur yang dianalisis secara tematik. Data data System Usability Dianalisis secara Kuantitatif

Menurut Miles dan Huberman (1994), analisis data kualitatif melibatkan tiga aktiviti utama yang berlaku secara serentak, iaitu:

\begin{enumerate}
  \item \textbf{Pengurangan Data (Data Reduction)} – Proses memilih, menumpukan, menyaring dan menstrukturkan data mentah ke dalam bentuk transkripsi yang bermakna.
  \item \textbf{Persembahan Data (Data Display)} – Penyusunan data dalam bentuk tema, jadual atau peta konsep untuk memudahkan interpretasi.
  \item \textbf{Pengesahan Data (Conclusion Drawing and Verification)} – Proses membuat kesimpulan dan mengesahkan dapatan melalui triangulasi dan semakan silang.
\end{enumerate}

Dalam kajian ini, transkrip temu bual dianalisis dan dikodkan mengikut tiga tema utama:

\begin{itemize}
  \item Aspek Teknologi
  \item Aspek Kepenggunaan
  \item Aspek Pedagogi
\end{itemize}

Analisis dokumen seperti rancangan pengajaran dan nota lapangan turut digunakan sebagai data sokongan untuk mengesahkan dapatan temu bual melalui kaedah triangulasi.

Menurut Marshall dan Rossman (1995), prosedur analisis data kualitatif terdiri daripada lima tahap utama:

\begin{enumerate}
  \item Mengorganisasi data
  \item Membaca dan menandakan nota awal
  \item Mengkategorikan data ke dalam tema
  \item Menyusun semula tema dan membina tafsiran
  \item Menyediakan laporan akhir
\end{enumerate}
Setiap peringkat dalam analisis data turut melibatkan proses pengurangan data (\textit{data reduction}) apabila penyelidik meneliti respons daripada responden semasa temu bual dijalankan. Kaedah temu bual dipilih kerana ia dianggap sebagai teknik terbaik untuk melaksanakan kajian kes secara intensif terhadap individu terpilih (Merriam, 2001). Dapatan data kualitatif ini memberikan pandangan mendalam daripada pelbagai perspektif dan berfungsi untuk menjelaskan serta mengukuhkan data kuantitatif yang tidak dapat dihuraikan melalui soal selidik (Yin, 1984; Ervin, 2005).

Ringkasan proses analisis data bagi fasa penilaian kepenggunaan ditunjukkan dalam Jadual~\ref{jadual:analisisPenilaian}.

\begin{table}[H]
\centering
\caption{Proses Analisis Data Fasa Penilaian Kepenggunaan}
\label{jadual:analisisPenilaian}
\begin{tabular}{|p{4cm}|p{5cm}|p{5cm}|}
\hline
\textbf{Perkara} & \textbf{Peringkat 1} & \textbf{Peringkat 2} \\
\hline
Instrumen & Temu bual separa struktur & Temu bual separa struktur \\
\hline
Analisis Data & Analisis Kuantitatif -Ujian Alfa & Analisis kualitatif, kesimpulan dan laporan berdasarkan Marshall \& Rossman \\
\hline
Tempoh Pelaksanaan & 1 minggu & 2 bulan \\
\hline
\end{tabular}
\end{table}
\subsection{Kesahan Instrumen dan Kesahan Kandungan Fasa Penilaian Kepenggunaan}

Data kualitatif dikumpulkan melalui dua set protokol temu bual separa struktur yang melibatkan pelajar dan pensyarah. Kesahan kriteria (\textit{criteria validity}) dan kesahan dalaman (\textit{internal validity}) digunakan untuk menilai kesahan instrumen temu bual.

Kesahan dalaman merujuk kepada penggunaan teknik \textit{independent rating} melalui kaedah \textit{cross-checking}, iaitu membandingkan maklumat yang diperoleh daripada responden pertama dengan responden kedua. Sekiranya maklumat yang diperoleh adalah konsisten dari segi isi, struktur dan teknik temu bual, maka instrumen tersebut dianggap mempunyai tahap kesahan yang tinggi.

Kesahan kriteria pula merujuk kepada penilaian yang dibuat terhadap individu yang mempunyai pengetahuan dan pengalaman dalam bidang kajian. Pendekatan ini memastikan bahawa data yang diperoleh adalah sah dan relevan dengan objektif kajian.
Penekanan turut diberikan kepada maklumat yang disampaikan oleh responden (Chua, 2011). Temu bual separa struktur dipilih kerana ia membolehkan penyelidik meneroka pemikiran dan persepsi responden terhadap sesuatu aktiviti secara mendalam (Patton, 2002). Menurut Cohen (2000), kaedah temu bual dapat mengukuhkan kefahaman terhadap fenomena yang dikaji.

Drever (1995) menyatakan bahawa pendekatan separa struktur mengekalkan objektif pengumpulan maklumat yang spesifik, namun tetap fleksibel dan sesuai dengan konteks responden. Temu bual separa struktur juga sesuai digunakan dalam penyelidikan kuantitatif dan kualitatif kerana ia menggabungkan ciri-ciri temu bual berstruktur dan tidak berstruktur (Mohd Nordin Abu Bakar, 2011).

Fasa ketiga kajian ini telah dijalankan secara sistematik mengikut urutan seperti yang ditunjukkan dalam Rajah~\ref{rajah:fasa3} di halaman~\pageref{rajah:fasa3}.
\section{Rumusan}

Bab ini membincangkan secara menyeluruh reka bentuk kajian yang merangkumi tiga fasa utama iaitu Fasa Analisis Keperluan, Fasa Reka Bentuk dan Pembangunan, serta Fasa Penilaian Kepenggunaan. Perbincangan turut meliputi prosedur kajian, populasi dan sampel, instrumen kajian, kaedah pengumpulan data, teknik analisis data, serta aspek kesahan dan kebolehpercayaan instrumen.

Setiap fasa telah dijalankan secara sistematik dan disokong oleh pendekatan kuantitatif dan kualitatif yang sesuai dengan objektif kajian. Jadual~\ref{jadual:instrumenRingkasan}, Jadual~\ref{jadual:pengumpulanData}, dan Jadual~\ref{jadual:sampelKajian} memaparkan ringkasan berkaitan instrumen kajian, kaedah pengumpulan data, dan sampel kajian yang terlibat.
\begin{table}[H]
\centering
\caption{Ringkasan Instrumen Kajian}
\label{jadual:instrumenKajian}
\begin{tabular}{|c|p{4cm}|p{5cm}|p{4cm}|}
\hline
\textbf{Bil} & \textbf{Fasa} & \textbf{Instrumen} & \textbf{Bentuk} \\
\hline
1 & Analisis Keperluan (Kajian Rintis) & Soal Selidik & Skala Likert 5 mata \\
\hline
2 & Analisis Keperluan & Soal Selidik \newline Soalan Temu Bual & Skala Likert 5 mata (diubah suai selepas kajian rintis) \newline Temu bual terbuka (4 orang pakar) \\
\hline
3 & Reka Bentuk & Soal Selidik Teknik Fuzzy Delphi & Skala Fuzzy 7 mata \newline 20 panel pakar \\
\hline
4 & Penilaian Kepenggunaan & Temu Bual Penilaian Model (Model TUP) & Temu bual separa struktur \\
\hline
\end{tabular}
\end{table}
\begin{table}[H]
\centering
\caption{Ringkasan Pengumpulan Data}
\label{jadual:pengumpulanData}
\begin{tabular}{|c|p{4cm}|p{5cm}|p{5cm}|}
\hline
\textbf{Bil} & \textbf{Fasa / Peringkat} & \textbf{Objektif} & \textbf{Teknik Pengumpulan} \\
\hline
1 & Analisis Keperluan (Kajian Rintis) & Menilai kefahaman dan kebolehgunaan instrumen soal selidik & Soal selidik kepada 30 pelajar prasekolah (guru pelatih) \\
\hline
2 & Analisis Keperluan & Mengenal pasti keperluan terhadap pembangunan aplikasi AR & Soal selidik kepada 50 guru prasekolah \newline Temu bual 4 orang pakar \\
\hline
3 & Reka Bentuk & Membina reka bentuk aplikasi AR Alphabets berdasarkan konsensus pakar & Teknik Fuzzy Delphi \newline 20 panel pakar \\
\hline
4 & Penilaian Kepenggunaan & Menilai kepenggunaan aplikasi AR Alphabets dalam konteks sebenar & Temu bual separa struktur kepada 8 guru dan 2 pensyarah \\
\hline
\end{tabular}
\end{table}
\begin{table}[H]
\centering
\caption{Ringkasan Sampel Kajian}
\label{jadual:sampelKajian}
\begin{tabular}{|c|p{4cm}|p{5cm}|p{4cm}|}
\hline
\textbf{Bil} & \textbf{Fasa} & \textbf{Sampel} & \textbf{Instrumen / Bentuk} \\
\hline
1 & Analisis Keperluan (Kajian Rintis) & 30 pelajar prasekolah (guru pelatih) & Soal Selidik \newline Skala Likert 5 mata \\
\hline
2 & Analisis Keperluan & 50 guru prasekolah \newline 4 orang pakar & Soal Selidik \newline Temu bual terbuka \\
\hline
3 & Reka Bentuk & 20 pakar: \newline - 6 pakar kurikulum \newline - 8 pakar m-Pembelajaran \newline - 6 pakar teknologi pendidikan & Soal Selidik Teknik Fuzzy Delphi \newline Skala Fuzzy 7 mata \\
\hline
4 & Penilaian Kepenggunaan & 10 responden: \newline - 8 guru prasekolah \newline - 2 pensyarah IPG & Temu bual separa struktur \newline Model TUP \\
\hline
\end{tabular}
\end{table}
