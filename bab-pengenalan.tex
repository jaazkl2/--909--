\chapter{Pengenalan}
\label{bab:pengenalan}



\section{Pendahuluan}

Teknologi Augmented Reality (AR) telah mengalami evolusi yang signifikan, terutamanya dalam sektor pendidikan, kerana kemampuannya menggabungkan elemen maya dan nyata dalam satu ruang interaktif yang mampu meningkatkan keberkesanan pembelajaran (Azuma, 1997; Valentyna Kovalenko,2022). Dalam pendidikan prasekolah, asas literasi dilihat sebagai tunjang utama dalam membentuk kemahiran asas membaca dan menulis yang kritikal untuk perkembangan kognitif kanak-kanak ( Cheng Zhaoet al,2024). Beberapa kajian menunjukkan bahawa penggunaan pendekatan pembelajaran berasaskan visual, auditori, dan interaktif --- seperti yang disokong oleh teknologi AR --- dapat memudahkan pemahaman konsep asas huruf serta meningkatkan motivasi dan penglibatan murid (Chen et al., 2020; Rahmawati et al., 2022; ; Radu,Keqin Li et al, 2024
\hspace{1cm}Pendekatan tradisional dalam pendidikan awal, seperti penggunaan buku teks, kad imbas, dan latihan bertulis, masih menjadi pilihan utama di kebanyakan prasekolah. Namun, kurangnya unsur interaktif dalam kaedah konvensional ini sering menyebabkan sebilangan murid menghadapi kesukaran dalam mengenal pasti dan menguasai huruf dengan berkesan (Rahmawati et al., 2022;). Sebaliknya, aplikasi teknologi AR memberikan peluang pembelajaran yang lebih dinamik dan menyeronokkan dengan membolehkan murid memvisualisasikan huruf dalam bentuk tiga dimensi (3D), mendengar sebutan yang betul, serta berinteraksi dengan animasi yang dapat memperkukuhkan pemahaman konsep (Gunalan et al., 2023; ). Oleh yang demikian, integrasi AR dalam proses pengajaran dan pembelajaran bukan sahaja dapat merangsang minat dan motivasi murid, tetapi juga meningkatkan daya ingatan serta penguasaan literasi awal melalui pendedahan visual, auditori, dan elemen interaktif (UNESCO, 2022; Chen et al., 2020;  Roghayeh Leila Barmaki ett al., 2024).

\hspace{1cm}Kajian ini memfokuskan kepada penilaian keberkesanan aplikasi AR Alphabets dalam membantu murid prasekolah mengenal huruf dengan cara yang lebih menyeronokkan dan interaktif, di samping meneliti kelebihan serta kekurangan penggunaan teknologi ini berbanding pendekatan tradisional seperti buku teks dan latihan bertulis (Kementerian Pendidikan Malaysia, 2015; Gunalan et al., 2023). Melalui integrasi elemen visual, audio, dan animasi dalam aplikasi AR, diharapkan pembelajaran huruf menjadi lebih efektif serta dapat meningkatkan pemahaman dan minat murid (Billinghurst \& Dünser, 2). Dapatan kajian lepas turut membuktikan bahawa penglibatan aktif murid dalam aktiviti pembelajaran yang menyeronokkan dan interaktif mampu mempercepat penguasaan literasi awal(Eric Yanchenko., 2024) . Sejajar dengan hasrat untuk memperkasakan pendidikan abad ke-21, Kementerian Pendidikan Malaysia (KPM) turut menyokong penggunaan teknologi inovatif seperti AR di peringkat prasekolah melalui Pelan Pembangunan Pendidikan Malaysia (PPPM) 2013--2025 (Kementerian Pendidikan Malaysia, 2013).

\hspace{1cm}Menurut Kementerian Pendidikan Malaysia (KPM), penggunaan teknologi AR dalam pendidikan prasekolah dapat membantu murid memahami bentuk serta bunyi huruf dengan lebih jelas melalui gabungan visualisasi dan audio yang interaktif. Pendekatan ini juga mampu merangsang perkembangan kemahiran kognitif dan motor halus melalui aktiviti sentuhan serta manipulasi objek digital, selain menggalakkan pembelajaran kendiri di mana kanak-kanak dapat meneroka huruf dalam suasana pembelajaran yang menyeronokkan dan motivasi tinggi (Kementerian Pendidikan Malaysia, 2015; Gunalan et al., 2023). Tambahan pula, laporan UNESCO MGIEP menegaskan bahawa teknologi AR dapat meningkatkan tumpuan dan daya ingatan murid kerana persembahan konsep secara visual dan interaktif mendorong pelajar untuk lebih mudah mengingati serta memahami maklumat (UNESCO, 2022)


\hspace{1cm}Kajian ini bertujuan membangunkan serta menilai aplikasi AR Alphabets dalam konteks pembelajaran prasekolah, di samping meneliti bagaimana teknologi inovatif ini dapat menyokong murid dalam mengenali dan memahami huruf secara lebih berkesan(Mukhlis Amien,2024). Kementerian Pendidikan Malaysia (KPM) telah mengambil langkah proaktif dalam mereformasi sistem pendidikan negara, termasuk menerusi pelaksanaan inisiatif seperti Mesyuarat Susulan Jemaah Menteri Bil. 6/2008 dan Pelan Pembangunan Pendidikan Malaysia (PPPM) 2013--2025, yang menekankan kepentingan penggunaan teknologi digital dalam pendidikan (Kementerian Pendidikan Malaysia, 2015). Sejajar dengan tuntutan Revolusi Industri Keempat (IR 4.0), KPM memperkenalkan Pendidikan 4.0 yang memberi fokus kepada penguasaan kemahiran abad ke-21, antaranya pemikiran kritis, kreativiti, penyelesaian masalah dan penerapan pembelajaran berasaskan teknologi (Kementerian Pendidikan Malaysia, 2015; UNESCO, 2022; Gunalan et al., 2023).

\hspace{1cm}Pengaplikasian teknologi seperti Augmented Reality (AR) semakin diiktiraf sebagai pemangkin pemodenan dalam pendidikan, di mana guru dapat menyampaikan kandungan pengajaran dengan lebih menarik, interaktif, dan efektif. Inisiatif ini selari dengan dasar ICT dalam pendidikan negara yang menekankan penggunaan teknologi digital sebagai strategi meningkatkan kualiti dan akses kepada pembelajaran (Kementerian Pendidikan Malaysia, 2015; Gunalan et al., 2023). Meskipun terdapat cabaran seperti kos pembangunan aplikasi AR serta keperluan latihan khusus untuk guru, teknologi ini tetap berpotensi besar dalam memperkasakan sistem pendidikan Malaysia agar lebih moden, responsif dan inklusif, selaras dengan aspirasi Pendidikan 4.0 (Gunalan et al., 2023; UNESCO, 2022).


\section{ Latar Belakang Kajian}

\vspace{1em}

Pembelajaran literasi awal merupakan asas utama dalam perkembangan akademik dan kognitif kanak-kanak, terutamanya bagi murid prasekolah yang sedang belajar mengenali huruf dan perkataan ( Zhou et al., 2021). Kajian menunjukkan bahawa kemahiran membaca dan menulis yang kukuh di peringkat prasekolah berkait rapat dengan prestasi akademik di sekolah rendah dan seterusnya (Zhou et al., 2021).

\hspace{1cm}Pendekatan tradisional dalam pembelajaran huruf, seperti penggunaan buku teks, kad imbas, dan kaedah pengulangan, masih digunakan dalam sistem pendidikan. Walau bagaimanapun, kaedah ini mungkin kurang menarik bagi murid prasekolah dan boleh menyebabkan mereka hilang fokus semasa belajar (Chen et al., 2020). Oleh itu, integrasi teknologi seperti Augmented Reality (AR) menawarkan pendekatan pembelajaran yang lebih interaktif dan sesuai dengan perkembangan digital dalam pendidikan awal kanak-kanak.

\hspace{1cm}Teknologi AR membolehkan pengguna berinteraksi dengan objek digital dalam dunia nyata, menjadikan pengalaman pembelajaran lebih visual, dinamik, dan menarik (Azuma, 1997). Kajian telah menunjukkan bahawa penggunaan AR dalam pendidikan mampu meningkatkan pemahaman murid, motivasi, dan daya ingatan, serta membolehkan mereka mengalami konsep pembelajaran dengan lebih realistik (Wu et al., 2013).

\hspace{1cm}Kajian oleh Ruihe Wang  (2024) mendapati bahawa murid yang belajar menggunakan modul interaktif berasaskan AR mampu mengingati konsep dengan lebih cepat berbanding mereka yang menggunakan bahan pembelajaran tradisional. Sementara itu, Rahmawati et al. (2022) menunjukkan bahawa AR dapat membantu kanak-kanak mengenali huruf dengan lebih efektif melalui penggunaan model 3D dan kesan animasi. Di Malaysia, KPM menggalakkan penggunaan AR dalam pendidikan prasekolah, selaras dengan usaha memperkukuh literasi digital generasi muda (KPM, 2013).

\hspace{1cm}Di Malaysia, Kementerian Pendidikan Malaysia (KPM) telah menggalakkan penggunaan teknologi digital dalam pendidikan prasekolah, termasuk elemen gamifikasi, AR, dan multimedia interaktif, sejajar dengan Pelan Pembangunan Pendidikan Malaysia (PPPM) 2013--2025.Menurut laporan KPM, penggunaan AR dalam pendidikan prasekolah boleh membantu murid untuk memahami bentuk dan bunyi huruf dengan lebih jelas melalui visualisasi dan audio interaktif, meningkatkan kemahiran kognitif dan motor halus dengan aktiviti sentuhan serta manipulasi objek huruf dalam AR, serta menggalakkan pembelajaran kendiri yang membolehkan kanak-kanak meneroka huruf secara lebih menyeronokkan.

\hspace{1cm}Selain itu, laporan UNESCO MGIEP menyatakan bahawa teknologi AR dapat meningkatkan tumpuan dan daya ingatan pelajar, kerana mereka lebih cenderung mengingati sesuatu konsep apabila ia dipersembahkan dalam bentuk visual dan interaktif (UNESCO, 2022).Berdasarkan maklumat di atas, kajian ini akan membangunkan dan menilai keberkesanan aplikasi AR Alphabets dalam membantu murid prasekolah mengenali huruf dengan pendekatan yang lebih interaktif dan menyeronokkan.Kajian ini juga akan menggunakan pengujian kebolehgunaan seperti System Usability Scale (SUS) untuk menilai kemudahan penggunaan aplikasi serta pengalaman pengguna.(Subash Neupane  et al.,2024)

\section{{Pernyataan Masalah}}

Pembelajaran literasi awal merupakan asas penting dalam pendidikan prasekolah kerana ia membantu kanak-kanak mengenali huruf, memahami bunyi(Kamyar Zeinalipour,2024) dan mengembangkan kemahiran membaca. Walaupun terdapat pelbagai inovasi teknologi pendidikan, kaedah konvensional masih menjadi pilihan utama di peringkat prasekolah. Kajian menunjukkan murid prasekolah menghadapi cabaran dalam pembelajaran literasi disebabkan kekurangan elemen interaktif dan motivasi yang rendah (Rahmawati et al., 2022).Guru juga menghadapi keterbatasan dari segi latihan penggunaan teknologi seperti AR (UNESCO, 2022). Justeru, pembangunan aplikasi AR Alphabets ini diharap dapat menambah nilai dan meningkatkan keberkesanan pembelajaran literasi awal.Beberapa isu utama yang dikenal pasti dalam pembelajaran huruf bagi kanak-kanak prasekolah adalah seperti berikut:\\


\begin{enumerate}[label=\roman*.]
            \item Kurangnya elemen interaktif dalam pembelajaran huruf.
            \item Motivasi pembelajaran yang rendah dalam kalangan murid prasekolah.
            \item Kesukaran mengingat bentuk dan bunyi huruf, terutama bagi huruf yang mempunyai bentuk hampir serupa (contoh: ``b'' dan ``d'').
            \item Keterbatasan guru dalam menerapkan teknologi pendidikan, kerana tidak semua guru diberi latihan yang mencukupi untuk menggunakan alat pembelajaran digital seperti AR (UNESCO, 2022).
        \end{enumerate}
    

\hspace{1cm}Kajian menunjukkan bahawa kaedah pembelajaran berasaskan visual dan auditori dapat membantu meningkatkan kefahaman murid (Anne-Flore Cabouat,2024). Murid prasekolah sering menghadapi cabaran dalam mengenal pasti bentuk huruf serta mengingati bunyi huruf dengan betul. Selain itu, kajian mendapati bahawa pelajar lebih cenderung untuk hilang fokus dalam pembelajaran huruf apabila tiada elemen interaktif dan menarik, menyebabkan mereka lambat dalam proses pengecaman huruf dan sebutan (Rahmawati et al., 2022).

\hspace{1cm}Sebagai penyelesaian kepada masalah ini, teknologi Augmented Reality (AR) menawarkan pendekatan pembelajaran yang lebih visual, interaktif, dan menarik(Mohammad Ali,2020) AR dapat membantu murid prasekolah melihat, mendengar, dan berinteraksi dengan huruf dalam bentuk 3D, menjadikan pengalaman pembelajaran lebih menyeronokkan dan mudah difahami.

\hspace{1cm}Kajian ini bertujuan untuk menilai keberkesanan aplikasi AR Alphabets dalam meningkatkan pengalaman pembelajaran huruf bagi murid prasekolah, serta mengenal pasti keuntungan dan cabaran teknologi ini berbanding kaedah pembelajaran tradisional. Murid prasekolah cenderung belajar dengan menggunakan deria mereka, namun kaedah pembelajaran tradisional kurang menawarkan visualisasi dinamik, animasi, dan elemen auditori yang dapat membantu mereka mengenali huruf dengan lebih berkesan (Chen et al., 2020).

\section{Objektif Kajian}

Kajian ini bertujuan untuk:
\begin{enumerate}[label=\roman*.]
    \item Mengenal pasti keperluan teknikal dan pedagogi dalam pembangunan aplikasi AR Alphabets untuk memfasilitasi pengenalan huruf dan fonetik kepada murid prasekolah.
    \item Merekabentuk dan membangunkan aplikasi AR Alphabets.
    \item Menentukan tahap kebolehgunaan dan penerimaan aplikasi AR Alphabets dalam kalangan murid serta guru sebagai alat sokongan pembelajaran literasi awal.
\end{enumerate}

\section{Soalan Kajian}
Persoalan kajian adalah seperti berikut:

\begin{enumerate}[label=\roman*.]
    \item Apakah keperluan teknikal  dalam pembangunan aplikasi \textit{AR Alphabets} untuk memfasilitasi pembelajaran pengenalan huruf dan fonetik kepada murid prasekolah?
    \item Bagaimanakah rekabentuk aplikasi \textit{AR Alphabets }l?
    \item Sejauh manakah tahap kebolehgunaan dan penerimaan aplikasi \textit{AR Alphabets }dalam kalangan murid serta guru sebagai alat sokongan pembelajaran literasi awal ?
\end{enumerate}
           

\section{Batasan Kajian}

Kajian ini menumpukan kepada penggunaan aplikasi \emph{AR Alphabets} dalam pembelajaran literasi awal bagi murid prasekolah di sebuah institusi pendidikan prasekolah yang dipilih.Struktur organisasi prasekolah yang menjadi tempat kajian terdiri daripada seorang guru besar, seorang penolong kanan pentadbiran, seorang penolong kanan hal ehwal murid, dan seorang penolong kanan kokurikulum. Institusi tersebut mempunyai jumlah keseluruhan murid prasekolah, dengan sekumpulan murid yang dipilih sebagai sampel kajian berdasarkan pengalaman mereka dalam pembelajaran literasi awal(Hari Prabhat Gupta, 2023) \\

\hspace{1cm}Kajian ini tidak melibatkan murid pendidikan khas, tetapi memberi tumpuan kepada murid prasekolah yang mengikuti kurikulum biasa, khususnya dalam pembelajaran mengenal huruf dan memahami bunyi fonetik menggunakan teknologi \emph{Augmented Reality (AR)}(Ashvini Varatharaj,2024) .Saiz sampel kajian terdiri daripada sekumpulan murid prasekolah yang dipilih berdasarkan interaksi mereka dengan \emph{AR Alphabets}, bagi menilai impak aplikasi terhadap pemahaman huruf, daya tumpuan, dan motivasi pembelajaran mereka.

\hspace{1cm}Kajian ini terhad kepada satu institusi prasekolah, dan penemuan yang diperoleh akan memberi gambaran tentang keberkesanan AR dalam pendidikan awal, namun tidak boleh digeneralisasikan kepada semua sekolah prasekolah di Malaysia tanpa kajian lanjut.

\section{Kepentingan Kajian}

Kajian ini mempunyai kepentingan yang besar dalam bidang pendidikan prasekolah, khususnya dalam pembelajaran literasi awal menggunakan teknologi \emph{Augmented Reality (AR)}.

\subsection{Kepentingan kepada Murid Prasekolah}
\begin{enumerate}[label=\roman*.]
    \item Meningkatkan pemahaman dan daya ingatan murid terhadap bentuk dan bunyi huruf melalui visualisasi interaktif.
    \item Menggalakkan pembelajaran kendiri, membolehkan murid berinteraksi dengan huruf dalam bentuk 3D dan memahami konsep secara aktif.
    \item Menjadikan pembelajaran lebih menyeronokkan dan menarik, membantu meningkatkan motivasi murid dalam mengenali huruf dengan lebih cepat 
\end{enumerate}

\subsection{Kepentingan kepada Guru}
\begin{enumerate}[label=\roman*.]
    \item Membantu guru dalam menyampaikan pelajaran dengan lebih efektif, menggunakan animasi 3D dan bunyi sebutan huruf.
    \item Menyediakan alat bantu mengajar yang inovatif, yang boleh digunakan untuk meningkatkan keberkesanan pengajaran literasi awal \cite{gunalan2023}.
    \item Memudahkan guru mengenal pasti kesulitan murid dalam pembelajaran huruf, dengan adanya sistem interaktif dan maklum balas digital.
\end{enumerate}

\subsection{Kepentingan kepada Sistem Pendidikan}
\begin{enumerate}[label=\roman*.]
    \item Menyokong Pelan Pembangunan Pendidikan Malaysia (PPPM) 2013-2025, yang menggalakkan penggunaan teknologi dalam pendidikan.
    \item Membantu memperkaya kurikulum pendidikan prasekolah, dengan mengintegrasikan teknologi digital dan pembelajaran interaktif \cite{kpm2013, unesco2022}.
    \item Menjadi rujukan kepada kajian teknologi pendidikan, khususnya dalam pengembangan aplikasi pembelajaran berbasis AR bagi kanak-kanak.
\end{enumerate}

\subsection{Kepentingan kepada Penyelidikan Teknologi}
\begin{enumerate}[label=\roman*.]jewak
    \item Menyumbang kepada inovasi teknologi pendidikan, dengan membangunkan aplikasi \emph{AR Alphabets} yang lebih mesra pengguna \cite{Patricia Piedade et., 2024}.
    \item Membantu dalam memahami keberkesanan AR dalam literasi awal, dengan menggunakan metodologi pengujian \emph{usability} seperti \emph{System Usability Scale (SUS)(Subash et al.,2024) 
    \item Menjadi asas kepada kajian lanjut dalam bidang AR, khususnya dalam pembangunan aplikasi pendidikan interaktif untuk murid prasekolah(Carlos Guerrero,2024)
\end{enumerate}

\section{Definisi Operasi}

\subsection{Augmented Reality (AR)}
\textbf{Definisi Umum:} Teknologi yang menggabungkan elemen digital ke dalam dunia nyata, membolehkan pengguna berinteraksi dengan objek maya dalam persekitaran fizikal (Prakash, 2025)

\textbf{Definisi Operasi dalam Kajian Ini:} AR digunakan dalam aplikasi \emph{AR Alphabets} untuk membantu murid prasekolah melihat, mendengar, dan berinteraksi dengan huruf dalam bentuk 3D bagi meningkatkan pemahaman mereka terhadap literasi awal.

\subsection{Literasi Awal}
\textbf{Definisi Umum:} Keupayaan kanak-kanak untuk mengenali huruf, memahami bunyi, dan mengembangkan kemahiran membaca serta menulis \cite{mayo2019}.

\textbf{Definisi Operasi dalam Kajian Ini:} Literasi awal merujuk kepada kemampuan murid prasekolah mengenali dan mengingat bentuk serta bunyi huruf, yang diuji melalui penggunaan aplikasi \emph{AR Alphabets}.

\subsection{Murid Prasekolah}
\textbf{Definisi Umum:} Kanak-kanak berusia 4 hingga 6 tahun yang berada dalam fasa pendidikan awal sebelum memasuki sekolah rendah \cite{unesco2022}.

\textbf{Definisi Operasi dalam Kajian Ini:} Murid prasekolah yang terlibat dalam kajian ini berusia 5 hingga 6 tahun, di mana mereka diuji untuk melihat keberkesanan penggunaan AR dalam pembelajaran huruf.

\subsection{Aplikasi Pembelajaran}
Aplikasi pembelajaran ialah perisian yang dibangunkan untuk menyokong proses pengajaran dan pembelajaran (PdP). Dalam kajian ini, aplikasi \emph{AR Alphabets} dibangunkan sebagai bahan bantu mengajar (BBM) dalam pembelajaran literasi awal, membolehkan murid mengimbas kad huruf dan melihat animasi interaktif sebagai sebahagian daripada kaedah pembelajaran digital yang lebih menarik.

\subsection{Pencapaian}
Pencapaian akademik dalam kajian ini merujuk kepada kemampuan murid mengenal huruf dan memahami fonetik selepas menggunakan \emph{AR Alphabets}. Kajian menilai kemajuan murid melalui ujian pra dan ujian pasca, bagi melihat sejauh mana aplikasi ini membantu mereka mengenal pasti huruf dengan lebih berkesan.



